\documentclass[11pt]{article}
\include{preamble} %% Separate file for preamble with macros and stuff


%% Titlea
\title{Laboration 1: PID-controls\\ {\small Sensors and Sensing}}
\author{Michael Flo{\ss}mann, Tom Olsson}
\date{\today}

\begin{document}
\maketitle %Title area
\listoffigures % List of all figures
\lstlistoflistings % List of all code snippets
\lstset{ matchrangestart=t} %initialise the linerange-macro for \lstinput...
\section{Theory and motivation}
\subsection{PID controller}
PID in the name PID-controller is short for \emph{P}roportional-\emph{Integral}-\emph{Derivative}-controller. As this implies, the controlling signal is based on a proportion of the current value, the previous values, and the rate of change of the observed value. The mathemathical formulation of this can be seen in \vref{eq:pid}.\par \vspace{10pt}
{\footnotesize
  \begin{tabular}{l l l}
    \textbf{Let:} \\
 &$e(t)$ &be some error measurement between current state and preferred state\\
 &$K_p$, $K_i$, $K_d$ &be the respective weights for the proportional, integral and derivate terms \\
 &$u(t)$ &be the output signal at time \emph{t} \\
    \textbf{Then:}
  \end{tabular}
  \begin{align}
    u(t) &= K_p\cdot e(t) + K_i \cdot\int_{0}^{t}e(\tau)\cdot \dif\tau + K_d \cdot \od{e(t)}{t}\label{eq:pid}
  \end{align}}
\par

\subsection{Minimum jerk}
The minimum jerk equation is an important part of creating smooth control. When a rotating actuator such as a motor starts, both the rotor and the stator will be at rest. The momentum generated by the motor can therefore cause movement in either part. As this can create an unwanted jerk while the rotor accelerates, it is important to accelerate slowly so that the stator remains at rest in relation to the reference frame. This can be achieved by the \emph{minimum jerk equation} shown in \vref{eq:mje}.
\par \vspace{10pt}
{\footnotesize
  \begin{tabular}{l l l}
    \textbf{Let:} \\
 &$x_i$, $x_f$ &be the initial and final states\\
 &$t$, $T$ &be the elapsed time since the action started, and the preferred total time for the action  \\
 &$x(t)$ &be the estimated state at time \emph{t} \\
    \textbf{Then:}
  \end{tabular}
  \begin{align}
    x(t) &= x_i +  (x_f - x_i) \cdot \left[10\left(\frac{t}{T}\right)^3 - 15\left(\frac{t}{T}\right)^4 + 6\left(\frac{t}{T}\right)^6 \right]\label{eq:mje}          
  \end{align}}

The $T$ parameter has to be estimated. If $T$ is much larger than  the actual time that is needed for the trajectory, the velocity will be very low, and if $T$ is too low $x(t)$ will approach infinity unless $\frac{t}{T}$ is clamped to $[0,1]$. However, this solution is not optimal. Instead, we choose to calculate the optimal time $T_{opt}$ as follows.%, and this can be seen in \ref{eq:derivative}.
\par
For finding out the optimal time $T_{opt}$, we substitute:
\begin{align}
  \label{eq:tau_substitution}
  \tau:=\frac{t}{T}
\end{align}
\begin{align}
\ref{eq:mje}\Rightarrow x(\tau)&= x_i +  (x_f - x_i) \cdot \left(10\tau^3 - 15\tau^4 + 6\tau^6 \right)\\
\od{x(\tau)}{\tau}&= (x_f - x_i)\cdot(30\tau^2-60\tau^3+36\tau^5)\label{eq:dx_dtau}
\end{align}
\ref{eq:dx_dtau} reaches its' maximum at $\tau=0.5$ (proof trivial) with the value:
\begin{align}
  \eval{\od{x(\tau)}{\tau}}_{\tau=0.5}&=\frac{15}{18}\cdot(x_f - x_i)
\end{align}
In order to make this term dependent on $T$, we must resubstitute:
\begin{align}
  \ref{eq:tau_substitution}\Rightarrow \od{\tau}{t}&=\frac{1}{T}\\
  \Rightarrow\dif\tau&=\dif t\cdot\frac{1}{T}\label{eq:dtau_resubs}\\
  \ref{eq:dx_dtau},\ref{eq:dtau_resubs}\Rightarrow\eval{T\cdot\od{x(\tau)}{t}}_{\tau=0.5}&=\frac{15}{18}\cdot(x_f - x_i)\\
  \Rightarrow\eval{\od{x(\tau)}{t}}_{\tau=0.5}&=\frac{15}{18}\cdot\frac{x_f - x_i}{T}\label{eq:dx_dt}
\end{align}
Now, if we state an optimal maximum velocity $v_{opt}$ for the minimum jerk equation, we can calculate the optimal time $T_{opt}$ for this velocity:
\begin{align}
  \Rightarrow\eval{\od{x(\tau)}{t}}_{\tau=0.5}&=v_{opt}\\
  \ref{eq:dx_dt}\Rightarrow T_{opt}&=\frac{15}{18}\cdot\frac{x_f - x_i}{v_{opt}}
\end{align}
\par

\section{Implementation}
% TODO add task description here

\subsection{Hardware}
The laboration is performed using an \emph{Arduino Due} microcontroller
\subsection{Position controller}
Describe the implementation of the position controller
\subsection{Velocity controller}
Describe the implementation of the velocity controller


\section{Verification and results}
\subsection{PID-tuning}
\subsection{Results}

\end{document}





%%% Local Variables:
%%% mode: latex
%%% TeX-master: t
%%% End:
