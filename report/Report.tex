\documentclass[11pt]{article}
\include{preamble} %% Separate file for preamble with macros and stuff


%% Titlea
\title{Laboration 1: PID-controls\\ {\small Sensors and Sensing}}
\author{Michael Flo{\ss}mann, Tom Olsson}
\date{\today}

\begin{document}
\maketitle %Title area
\listoffigures % List of all figures
\lstlistoflistings % List of all code snippets
\lstset{ matchrangestart=t} %initialise the linerange-macro for \lstinput...

\section{Implementation of a PID controller}
PID in the name PID-controller is short for \emph{P}roportional-\emph{Integral}-\emph{Derivative}-controller. As this implies, the controlling signal is based on a proportion of the current value, the previous values, and the rate of change of the observed value. The mathemathical formulation of this can be seen in \vref{eq:pid}.\par \vspace{10pt}
{\footnotesize
  \begin{tabular}{l l l}
    \textbf{Let:} \\
 &$e(t)$ &be some error measurement between current state and preferred state\\
 &$K_p$, $K_i$, $K_d$ &be the respective weights for the proportional, integral and derivate terms \\
 &$u(t)$ &be the output signal at time \emph{t} \\
    \textbf{Then:}
  \end{tabular}
  \begin{align}
    u(t) &= K_p\cdot e(t) + K_i \cdot\int_{0}^{\infty}e(t)\cdot dt + K_d \cdot \frac{de}{dt}\label{eq:pid}
  \end{align}}
\par

\subsection{Minimum jerk}
The minimum jerk equation is an important part of creating smooth control. When a rotating actuator such as a motor starts, both the rotor and the stator will be at rest. The momentum generated by the motor can therefore cause movement in either part. As this can create an unwanted jerk while the rotor accelerates, it is important to accelerate slowly so that the stator remains at rest in relation to the reference frame. This can be achieved by the \emph{minimum jerk equation} shown in \vref{eq:mje}.
\par \vspace{10pt}
{\footnotesize
  \begin{tabular}{l l l}
    \textbf{Let:} \\
 &$x_i$, $x_f$ &be the initial and final states\\
 &$t$, $T$ &be the elapsed time since the action started, and the preferred total time for the action  \\
 &$x(t)$ &be the estimated state at time \emph{t} \\
    \textbf{Then:}
  \end{tabular}
  \begin{align}
    x(t) &= x_i +  (x_f - x_i) \cdot \left[10\left(\frac{t}{T}\right)^4 \right]\label{eq:mje}
  \end{align}}
\par

\section{Position controller}
Describe the implementation of the position controller
\section{Velocity controller}
Describe the implementation of the velocity controller
\end{document}





%%% Local Variables:
%%% mode: latex
%%% TeX-master: t
%%% End:
