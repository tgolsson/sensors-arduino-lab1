\documentclass[11pt]{article}
\usepackage{graphicx}
\usepackage{caption}
\usepackage[a4paper, top=1in, bottom=1.1in, left=1in, right=1in]{geometry}
\usepackage[utf8]{inputenc} % utf8
\usepackage[T1]{fontenc}
\usepackage{xcolor}
\usepackage{listings}
\usepackage{subcaption}
\usepackage{siunitx}\usepackage{wrapfig}
\usepackage{varioref}\usepackage{amsmath}
\usepackage{commath}


\setlength{\belowcaptionskip}{-6pt}
\makeatletter
\lst@Key{matchrangestart}{f}{\lstKV@SetIf{#1}\lst@ifmatchrangestart}
\def\lst@SkipToFirst{%
  \lst@ifmatchrangestart\c@lstnumber=\numexpr-1+\lst@firstline\fi
  \ifnum \lst@lineno<\lst@firstline
  \def\lst@next{\lst@BeginDropInput\lst@Pmode
    \lst@Let{13}\lst@MSkipToFirst
    \lst@Let{10}\lst@MSkipToFirst}%
  \expandafter\lst@next
  \else
  \expandafter\lst@BOLGobble
  \fi}
\makeatother

\lstset{  
  backgroundcolor=\color{gray!30},   % choose the background color; you must add \usepackage{color} or \usepackage{xcolor}
  basicstyle=\footnotesize,        % the size of the fonts that are used for the code
  breakatwhitespace=false,         % sets if automatic breaks should only happen at whitespace
  breaklines=true,                 % sets automatic line breaking
  captionpos=t,                    % sets the caption-position to bottom
  escapeinside={\%*}{*)},          % if you want to add LaTeX within your code
  extendedchars=true,              % lets you use non-ASCII characters; for 8-bits encodings only, does not work with UTF-8
  frame=single,                   % adds a frame around the code
  keepspaces=true,                 % keeps spaces in text, useful for keeping indentation of code (possibly needs columns=flexible)
  keywordstyle=\color{blue},       % keyword style
  language=C++,                 % the language of the code
  numbers=left,                    % where to put the line-numbers; possible values are (none, left, right)
  numbersep=20pt,                   % how far the line-numbers are from the code
  numberstyle=\tiny\color{gray}, % the style that is used for the line-numbers
  rulecolor=\color{blue!20},       
  showspaces=false,                % show spaces everywhere adding particular underscores; it overrides 'showstringspaces'
  showstringspaces=false,          % underline spaces within strings only
  showtabs=false,                  % show tabs within strings adding particular underscores
  stepnumber=1,                    % the step between two line-numbers. If it's 1, each line will be numbered
  tabsize=2,                   % sets default tabsize to 2 spaces
  language=Octave,
  framesep=7pt,
  xleftmargin=12pt,
  xrightmargin=11pt
}

\setlength{\fboxsep}{4pt}
\DeclareCaptionFormat{myformat}{%
  \hspace{1pt}\fcolorbox{blue!20}{gray!20}{\footnotesize\parbox{\dimexpr\textwidth-17pt\fboxsep\fboxrule\relax}{#1#2\ttfamily#3}}\vspace{-4pt}
}
\captionsetup[lstlisting]{format=myformat}
\captionsetup[figure]{labelfont=sf,hypcap=false,format=hang,margin=0.5cm,justification=RaggedRight,calcwidth=0.7\linewidth,font=footnotesize,justification=justified}
\captionsetup[table]{labelfont=sf,hypcap=false,format=hang,margin=1cm,justification=RaggedRight,calcwidth=0.8\linewidth,font=footnotesize,justification=justified}
\labelformat{equation}{(#1)}

%%% Local Variables:
%%% mode: latex
%%% TeX-master: t
%%% End:

%%% Math typesetting macros
\newcommand{\di}[2]{#1_\textup{#2}} % Descriptive Index: Macro for quick upright index (as opposed to a variable index, which should be italic) %% Separate file for preamble with macros and stuff


%% Titlea
\title{Laboration 1: PID-controls\\ {\small Sensors and Sensing}}
\author{Michael Flo{\ss}mann, Tom Olsson}
\date{\today}

\begin{document}
\maketitle %Title area
\listoffigures % List of all figures
\lstlistoflistings % List of all code snippets
\lstset{ matchrangestart=t} %initialise the linerange-macro for \lstinput...

\section{Implementation of a PID controller}
PID in the name PID-controller is short for \emph{P}roportional-\emph{Integral}-\emph{Derivative}-controller. As this implies, the controlling signal is based on a proportion of the current value, the previous values, and the rate of change of the observed value. The mathemathical formulation of this can be seen in \vref{eq:pid}.\par \vspace{10pt}
{\footnotesize
  \begin{tabular}{l l l}
    \textbf{Let:} \\
 &$e(t)$ &be some error measurement between current state and preferred state\\
 &$K_p$, $K_i$, $K_d$ &be the respective weights for the proportional, integral and derivate terms \\
 &$u(t)$ &be the output signal at time \emph{t} \\
    \textbf{Then:}
  \end{tabular}
  \begin{align}
    u(t) &= K_p\cdot e(t) + K_i \cdot\int_{0}^{\infty}e(t)\cdot dt + K_d \cdot \frac{de}{dt}\label{eq:pid}
  \end{align}}
\par

\subsection{Minimum jerk}
The minimum jerk equation is an important part of creating smooth control. When a rotating actuator such as a motor starts, both the rotor and the stator will be at rest. The momentum generated by the motor can therefore cause movement in either part. As this can create an unwanted jerk while the rotor accelerates, it is important to accelerate slowly so that the stator remains at rest in relation to the reference frame. This can be achieved by the \emph{minimum jerk equation} shown in \vref{eq:mje}.
\par \vspace{10pt}
{\footnotesize
  \begin{tabular}{l l l}
    \textbf{Let:} \\
 &$x_i$, $x_f$ &be the initial and final states\\
 &$t$, $T$ &be the elapsed time since the action started, and the preferred total time for the action  \\
 &$x(t)$ &be the estimated state at time \emph{t} \\
    \textbf{Then:}
  \end{tabular}
  \begin{align}
    x(t) &= x_i +  (x_f - x_i) \cdot \left[10\left(\frac{t}{T}\right)^3 - 15\left(\frac{t}{T}\right)^4 + 6\left(\frac{t}{T}\right)^6 \right]\label{eq:mje}
  \end{align}}
\par

\section{Position controller}
Describe the implementation of the position controller
\section{Velocity controller}
Describe the implementation of the velocity controller

\end{document}





%%% Local Variables:
%%% mode: latex
%%% TeX-master: t
%%% End:
